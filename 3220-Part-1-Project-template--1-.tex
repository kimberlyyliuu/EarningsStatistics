% Options for packages loaded elsewhere
\PassOptionsToPackage{unicode}{hyperref}
\PassOptionsToPackage{hyphens}{url}
\PassOptionsToPackage{dvipsnames,svgnames,x11names}{xcolor}
%
\documentclass[
  12pt,
]{article}
\usepackage{amsmath,amssymb}
\usepackage{iftex}
\ifPDFTeX
  \usepackage[T1]{fontenc}
  \usepackage[utf8]{inputenc}
  \usepackage{textcomp} % provide euro and other symbols
\else % if luatex or xetex
  \usepackage{unicode-math} % this also loads fontspec
  \defaultfontfeatures{Scale=MatchLowercase}
  \defaultfontfeatures[\rmfamily]{Ligatures=TeX,Scale=1}
\fi
\usepackage{lmodern}
\ifPDFTeX\else
  % xetex/luatex font selection
\fi
% Use upquote if available, for straight quotes in verbatim environments
\IfFileExists{upquote.sty}{\usepackage{upquote}}{}
\IfFileExists{microtype.sty}{% use microtype if available
  \usepackage[]{microtype}
  \UseMicrotypeSet[protrusion]{basicmath} % disable protrusion for tt fonts
}{}
\makeatletter
\@ifundefined{KOMAClassName}{% if non-KOMA class
  \IfFileExists{parskip.sty}{%
    \usepackage{parskip}
  }{% else
    \setlength{\parindent}{0pt}
    \setlength{\parskip}{6pt plus 2pt minus 1pt}}
}{% if KOMA class
  \KOMAoptions{parskip=half}}
\makeatother
\usepackage{xcolor}
\usepackage[margin=1in]{geometry}
\usepackage{color}
\usepackage{fancyvrb}
\newcommand{\VerbBar}{|}
\newcommand{\VERB}{\Verb[commandchars=\\\{\}]}
\DefineVerbatimEnvironment{Highlighting}{Verbatim}{commandchars=\\\{\}}
% Add ',fontsize=\small' for more characters per line
\usepackage{framed}
\definecolor{shadecolor}{RGB}{248,248,248}
\newenvironment{Shaded}{\begin{snugshade}}{\end{snugshade}}
\newcommand{\AlertTok}[1]{\textcolor[rgb]{0.94,0.16,0.16}{#1}}
\newcommand{\AnnotationTok}[1]{\textcolor[rgb]{0.56,0.35,0.01}{\textbf{\textit{#1}}}}
\newcommand{\AttributeTok}[1]{\textcolor[rgb]{0.13,0.29,0.53}{#1}}
\newcommand{\BaseNTok}[1]{\textcolor[rgb]{0.00,0.00,0.81}{#1}}
\newcommand{\BuiltInTok}[1]{#1}
\newcommand{\CharTok}[1]{\textcolor[rgb]{0.31,0.60,0.02}{#1}}
\newcommand{\CommentTok}[1]{\textcolor[rgb]{0.56,0.35,0.01}{\textit{#1}}}
\newcommand{\CommentVarTok}[1]{\textcolor[rgb]{0.56,0.35,0.01}{\textbf{\textit{#1}}}}
\newcommand{\ConstantTok}[1]{\textcolor[rgb]{0.56,0.35,0.01}{#1}}
\newcommand{\ControlFlowTok}[1]{\textcolor[rgb]{0.13,0.29,0.53}{\textbf{#1}}}
\newcommand{\DataTypeTok}[1]{\textcolor[rgb]{0.13,0.29,0.53}{#1}}
\newcommand{\DecValTok}[1]{\textcolor[rgb]{0.00,0.00,0.81}{#1}}
\newcommand{\DocumentationTok}[1]{\textcolor[rgb]{0.56,0.35,0.01}{\textbf{\textit{#1}}}}
\newcommand{\ErrorTok}[1]{\textcolor[rgb]{0.64,0.00,0.00}{\textbf{#1}}}
\newcommand{\ExtensionTok}[1]{#1}
\newcommand{\FloatTok}[1]{\textcolor[rgb]{0.00,0.00,0.81}{#1}}
\newcommand{\FunctionTok}[1]{\textcolor[rgb]{0.13,0.29,0.53}{\textbf{#1}}}
\newcommand{\ImportTok}[1]{#1}
\newcommand{\InformationTok}[1]{\textcolor[rgb]{0.56,0.35,0.01}{\textbf{\textit{#1}}}}
\newcommand{\KeywordTok}[1]{\textcolor[rgb]{0.13,0.29,0.53}{\textbf{#1}}}
\newcommand{\NormalTok}[1]{#1}
\newcommand{\OperatorTok}[1]{\textcolor[rgb]{0.81,0.36,0.00}{\textbf{#1}}}
\newcommand{\OtherTok}[1]{\textcolor[rgb]{0.56,0.35,0.01}{#1}}
\newcommand{\PreprocessorTok}[1]{\textcolor[rgb]{0.56,0.35,0.01}{\textit{#1}}}
\newcommand{\RegionMarkerTok}[1]{#1}
\newcommand{\SpecialCharTok}[1]{\textcolor[rgb]{0.81,0.36,0.00}{\textbf{#1}}}
\newcommand{\SpecialStringTok}[1]{\textcolor[rgb]{0.31,0.60,0.02}{#1}}
\newcommand{\StringTok}[1]{\textcolor[rgb]{0.31,0.60,0.02}{#1}}
\newcommand{\VariableTok}[1]{\textcolor[rgb]{0.00,0.00,0.00}{#1}}
\newcommand{\VerbatimStringTok}[1]{\textcolor[rgb]{0.31,0.60,0.02}{#1}}
\newcommand{\WarningTok}[1]{\textcolor[rgb]{0.56,0.35,0.01}{\textbf{\textit{#1}}}}
\usepackage{graphicx}
\makeatletter
\def\maxwidth{\ifdim\Gin@nat@width>\linewidth\linewidth\else\Gin@nat@width\fi}
\def\maxheight{\ifdim\Gin@nat@height>\textheight\textheight\else\Gin@nat@height\fi}
\makeatother
% Scale images if necessary, so that they will not overflow the page
% margins by default, and it is still possible to overwrite the defaults
% using explicit options in \includegraphics[width, height, ...]{}
\setkeys{Gin}{width=\maxwidth,height=\maxheight,keepaspectratio}
% Set default figure placement to htbp
\makeatletter
\def\fps@figure{htbp}
\makeatother
\setlength{\emergencystretch}{3em} % prevent overfull lines
\providecommand{\tightlist}{%
  \setlength{\itemsep}{0pt}\setlength{\parskip}{0pt}}
\setcounter{secnumdepth}{-\maxdimen} % remove section numbering
\usepackage{setspace}
\usepackage{ulem}
\usepackage{hyperref}
\onehalfspacing
\ifLuaTeX
  \usepackage{selnolig}  % disable illegal ligatures
\fi
\IfFileExists{bookmark.sty}{\usepackage{bookmark}}{\usepackage{hyperref}}
\IfFileExists{xurl.sty}{\usepackage{xurl}}{} % add URL line breaks if available
\urlstyle{same}
\hypersetup{
  pdftitle={STAT 3220 Group Project},
  pdfauthor={KHL},
  colorlinks=true,
  linkcolor={Maroon},
  filecolor={Maroon},
  citecolor={Blue},
  urlcolor={black},
  pdfcreator={LaTeX via pandoc}}

\title{STAT 3220 Group Project}
\author{KHL}
\date{}

\begin{document}
\maketitle

\newpage

\hypertarget{pledege}{%
\section{Pledege}\label{pledege}}

Please type your names in the appropriate space below. Failing to do so
will result in a 0 on this assignment.

``We have neither given nor received unauthorized help on this
assignment''

\begin{itemize}
\tightlist
\item
  Member 1: Kimberly Liu
\item
  Member 2: Hadley McQuerrey
\item
  Member 3: Luke Scheuer
\end{itemize}

\newpage

\hypertarget{background}{%
\subsection{Background}\label{background}}

\begin{itemize}
  \item Does a higher level of educational attainment generally increase personal earnings income across different states?
  \item Does personal earnings increase with an individual's health?
  \item: Do older individuals generally earn more money than younger individuals? 
\end{itemize}

The Census is beyond a historical record or constitutional mandate. The
decennial census plays a pivotal role in shaping the political and
economic landscape of the nation. Through the meticulous surveying of
every resident, the census results directly influence the allocation of
hundreds of billions of dollars in federal funding to states, counties,
and communities. This funding, critical for schools, hospitals, roads,
and public works, is determined by population totals and breakdowns by
sex, age, race, and other factors, ensuring that each community receives
its fair share based on its specific needs (Bureau, 2021). Beyond
funding, census data are instrumental in the equitable distribution of
resources for over 100 federal programs, including Medicaid, Head Start,
and SNAP. The accuracy of these counts ensures that funding is justly
allocated, supporting health, education, housing, and infrastructure
programs vital for community well-being.

Redistricting and apportionment are other critical areas where census
results hold significant sway. Post-census, state and local officials
utilize the updated data to redraw congressional, state, and local
district boundaries, ensuring that each district contains roughly equal
numbers of people (Mather \& Scommegna, 2019). This process is
fundamental to maintaining the one-person, one-vote rule, safeguarding
the equity of voting power across the nation. Additionally, the
apportionment of seats in the U.S. House of Representatives is based on
state population counts from the census (Farley, 2020). This means no
state has a permanent claim to its current number of House seats;
rather, these seats are redistributed according to each state's share of
the national population. This redistribution can shift political power,
with states experiencing population growth, often in the southern and
western regions, gaining seats at the expense of those in the Northeast
and Midwest. The initial data from the census, therefore, not only
impacts the immediate next decade of federal funding, redistricting, and
apportionment but also sets the stage for the political dynamics and
resource allocation of the future.

In essence, the decennial census results provide a foundation for a fair
and functional democracy, guiding crucial decisions on federal funding,
redistricting, and apportionment. These activities, in turn, influence
every aspect of American life, from local governance and community
services to the national political landscape, underscoring the
importance of accurate and comprehensive census data.

\hypertarget{data-description}{%
\subsection{Data Description}\label{data-description}}

The dataset was compiled from multiple reputable sources, primarily
focusing on the United States demographic, economic, and educational
landscapes as of 2020. The core of this dataset originates from the
Annual Social and Economic Supplement (ASEC) survey, conducted by the US
Census Bureau. This survey incorporates the basic Current Population
Survey (CPS)---a vital source for official government statistics on
employment and unemployment---alongside supplemental questions that
delve into poverty, geographic mobility/migration, and work experience.
To enrich this dataset further, unemployment rate data were incorporated
from the US Bureau of Labor Statistics, a key authority on labor market
activity and working conditions in the US. Additionally, the dataset
includes urban population percentages for each state, derived from the
US Census Bureau's Decennial Census of Population and Housing, providing
insights into urban versus rural demographics. Sales tax rates by state,
as reported by the Tax Foundation, were also integrated to offer a
financial perspective.

In preparing this comprehensive dataset, a merge process was used to
align the data by State. Given the ASEC survey's extensive
individual-level data, averages for variables such as highest level of
education attained, sex, total income, and age were calculated and then
aggregated on a State basis. To ensure consistency and relevance, the
dataset was cleaned to include only individuals aged 18 and above,
thereby aligning the education and income data more accurately with the
adult population. This curated ASEC data was subsequently merged with
state-sorted unemployment rate data from the Bureau of Labor Statistics
and sales tax rates from the Tax Foundation, ensuring a cohesive dataset
that facilitates multifaceted analyses.

During the data preparation process, several potential issues were
considered. The exclusion of individuals under 18 could limit insights
into the full spectrum of educational engagement and early income
patterns. Furthermore, while merging datasets from various sources
enhances the dataset's depth, it also introduces complexities related to
data compatibility and consistency. Nonetheless, the sources of this
data---namely the US Census Bureau, the Bureau of Labor Statistics, and
the Tax Foundation---are renowned for their reliability and the rigorous
methodologies they employ, underpinning the overall trustworthiness of
the dataset. This merged dataset, while primarily representing a
snapshot of 2020, offers a valuable foundation for exploring
socio-economic dynamics across the United States.

\hypertarget{exploratory-data-analysis}{%
\subsection{Exploratory Data Analysis}\label{exploratory-data-analysis}}

First, the Histogram of Personal Earnings displays the spread Outline -
must include histogram of response variable (total income) - must
include at least four graphical representations and the report should
address major features of the data. - You must include at least one
summary that relate to each of your research questions in some way

Graphs you can possibly include: - Histogram of total income (PEARNVAL)
- Scatterplot of total income (PEARNVAL) by highest education level
(H\_ED) - Scatterplot of Average Total Income by Gender across Different
Urbanization Levels - Scatterplot of total income (PEARNVAL) by age -
Boxplot or Scatterplot of total income (PEARNVAL) by health status (HEA)

\begin{Shaded}
\begin{Highlighting}[]
\NormalTok{\#\#\# Histogram of PEARNVAL}
\NormalTok{ggplot(incomeData,aes(x=PEARNVAL)) +}
\NormalTok{  geom\_histogram(bins=20)+}
\NormalTok{  labs(x="Personal Earnings (USD)",y="Count",title="Distribution of Personal Earnings by State") +}
\NormalTok{  theme\_linedraw()}
\end{Highlighting}
\end{Shaded}

\includegraphics{3220-Part-1-Project-template--1-_files/figure-latex/unnamed-chunk-2-1.pdf}

\begin{Shaded}
\begin{Highlighting}[]
\NormalTok{\#\#\#  Scatterplot of personal income (PEARNVAL) vs highest education level (H\_ED)}
\NormalTok{incomeData$Education\_Level \textless{}{-} cut(incomeData$H\_ED,}
\NormalTok{                            breaks = c(39, 40, 41, 42, 43),}
\NormalTok{                            labels = c("Some College but No Degree",}
\NormalTok{                                       "Associate Degree {-} Occupational/Vocational",}
\NormalTok{                                       "Associate Degree {-} Academic Program",}
\NormalTok{                                       "Bachelor\textquotesingle{}s Degree"),}
\NormalTok{                            include.lowest = TRUE)}

\NormalTok{ggplot(incomeData, aes(x = H\_ED, y = PEARNVAL, color = Education\_Level)) +}
\NormalTok{  geom\_point() +}
\NormalTok{  scale\_color\_manual(values = c("High School Graduate" = "green",}
\NormalTok{                                "Some College but No Degree" = "lightpink",}
\NormalTok{                                "Associate Degree {-} Occupational/Vocational" = "blue",}
\NormalTok{                                "Associate Degree {-} Academic Program" = "orange",}
\NormalTok{                                "Bachelor\textquotesingle{}s Degree" = "black")) +}
\NormalTok{  labs(x = "Highest Education (Numeric)", y = "Personal Earnings (USD)", color = "Educational Attainment", title = "Educational Attainment vs Personal Earnings") +}
\NormalTok{  theme(axis.text.x = element\_text(angle = 45, hjust = 1))}
\end{Highlighting}
\end{Shaded}

\includegraphics{3220-Part-1-Project-template--1-_files/figure-latex/unnamed-chunk-2-2.pdf}

\begin{Shaded}
\begin{Highlighting}[]
\NormalTok{\#\#\# Scatterplot of total income (PEARNVAL) by age}
\NormalTok{agecor\_coefficient \textless{}{-} cor(incomeData$AGE, incomeData$PEARNVAL, use = "complete.obs")}
\NormalTok{ggplot(incomeData, aes(x = AGE, y = PEARNVAL)) +}
\NormalTok{  geom\_point() +}
\NormalTok{  labs(x = "Age", y = "Personal Earnings", title = "Age vs Personal Earnings") +}
\NormalTok{  annotate("text", x = Inf, y = Inf, label = paste("r = ", round(agecor\_coefficient, 3)),}
\NormalTok{           hjust = 1.05, vjust = 2, size = 5, color = "black") +}
\NormalTok{  theme(axis.text.x = element\_text(angle = 45, hjust = 1),}
\NormalTok{        plot.title = element\_text(hjust = 0.5))}
\end{Highlighting}
\end{Shaded}

\includegraphics{3220-Part-1-Project-template--1-_files/figure-latex/unnamed-chunk-2-3.pdf}

\begin{Shaded}
\begin{Highlighting}[]
\NormalTok{\#\#\#Scatterplot of total income (PEARNVAL) by health status (HEA) }
\NormalTok{incomeData$Health\_Status\textless{}{-}cut(incomeData$HEA,}
\NormalTok{                              breaks= c(0,1,2,3,4,5),}
\NormalTok{                              labels= c("Excellent Health",}
\NormalTok{                                        "Very Good Health",}
\NormalTok{                                        "Good Health",}
\NormalTok{                                        "Fair Health",}
\NormalTok{                                        "Poor Health"))}
\NormalTok{heacor\_coefficient\textless{}{-}cor(incomeData$HEA,incomeData$PEARNVAL)}
\NormalTok{ggplot(incomeData, aes(x=HEA,y=PEARNVAL,color=Health\_Status)) + }
\NormalTok{  geom\_point() + }
\NormalTok{  scale\_color\_manual(values=c("Excellent Health" = "purple",}
\NormalTok{                              "Very Good Health" = "blue",}
\NormalTok{                              "Good Health" = "orange",}
\NormalTok{                              "Fair Health" = "black",}
\NormalTok{                              "Poor Health" = "deepskyblue4")) +}
\NormalTok{  labs(x="Health Status (Numeric)",y="Personal Earnings (USD)",color="Health Status",title="Health Status vs Personal Earnings") +}
\NormalTok{  theme\_linedraw() +}
\NormalTok{  annotate("text", x=Inf, y=Inf, label=paste("r = ", round(heacor\_coefficient,3)), }
\NormalTok{           hjust=1.1, vjust=2, size=5, color="black")}
\end{Highlighting}
\end{Shaded}

\includegraphics{3220-Part-1-Project-template--1-_files/figure-latex/unnamed-chunk-2-4.pdf}

\begin{Shaded}
\begin{Highlighting}[]
\NormalTok{\#\#\# Scatterplot of Average Total Income by Gender across Different Urbanization Levels}
\NormalTok{incomeData$Majority\_Sex\textless{}{-}cut(incomeData$SEX,}
\NormalTok{                             breaks=c(0,1.5,2),}
\NormalTok{                             labels=c("Male","Female"))}
\NormalTok{ggplot(incomeData,aes(x=URB\_PER,y=PEARNVAL,color=Majority\_Sex)) +}
\NormalTok{  geom\_point() +}
\NormalTok{  scale\_color\_manual(values=c("Male" = "blue",}
\NormalTok{                              "Female" = "hotpink")) +}
\NormalTok{  labs(x="Percentage of Urban Population",y="Personal Earnings (USD)",color="Majority Gender",title="\% Urban Population vs Personal Earnings Grouped by Majority Gender")}
\end{Highlighting}
\end{Shaded}

\includegraphics{3220-Part-1-Project-template--1-_files/figure-latex/unnamed-chunk-2-5.pdf}

\includegraphics{3220-Part-1-Project-template--1-_files/figure-latex/unnamed-chunk-3-1.pdf}
\includegraphics{3220-Part-1-Project-template--1-_files/figure-latex/unnamed-chunk-3-2.pdf}
\includegraphics{3220-Part-1-Project-template--1-_files/figure-latex/unnamed-chunk-3-3.pdf}

\hypertarget{conclusion}{%
\subsection{Conclusion}\label{conclusion}}

Write up a conclusion that addresses the following: - About 1-1.5 page
long

Outline: - interpret the graphical summaries - make sure to tie these
all to our 3 research questions - include a numerical summary that
corresponds with the graphical summary - include context if there are
any large discrepancies - use the code chunk above to compute any
summary statistics that you want to use in your conclusions. - include a
description as to why your response variable is suitable for regression.
If it is not suitable, you should discuss what you will need to do to
make it suitable

\newpage

\hypertarget{appendix-a-data-dictionary}{%
\subsection{Appendix A: Data
Dictionary}\label{appendix-a-data-dictionary}}

State \textbar{} State \textbar{} Desc \textbar{}\\
State by FIPS Code \textbar{} STATE \textbar{} Desc \textbar{}\\
Highest Education \textbar{} H\_ED \textbar{} Desc \textbar{}\\
Gender \textbar{} SEX \textbar{} Desc \textbar{}\\
Total Household Income \textbar{} TOT\_IN \textbar{} Desc \textbar{}\\
Total Personal Earnings \textbar{} PEARNVAL \textbar{} Desc \textbar{}\\
Health Status \textbar{} HEA \textbar{} Desc \textbar{}\\
Unemployment Rate \textbar{} UNEMP\_RATE \textbar{} Desc \textbar{}\\
Sales Tax Rate \textbar{} TAX\_RTE \textbar{} Desc \textbar{}\\
Percentage of Urban Residents \textbar{} URB\_PER \textbar{} Desc
\textbar{}

\newpage

\hypertarget{appendix-b-data-rows}{%
\subsection{Appendix B: Data Rows}\label{appendix-b-data-rows}}

\begin{verbatim}
   STATE       State     H_ED      SEX   TOT_IN PEARNVAL      HEA      AGE
1      1     Alabama 40.75434 1.483382 20.47182 53905.05 2.050578 42.60043
2      2      Alaska 40.93541 1.489043 22.22261 59908.18 2.087659 43.33103
3      4     Arizona 40.56492 1.475323 19.96128 54509.31 2.077449 41.63326
4      5    Arkansas 40.80033 1.491749 19.71452 53513.54 2.139439 43.28300
5      6  California 40.65105 1.456628 21.42030 62824.72 2.059366 42.26563
6      8    Colorado 41.16509 1.457413 22.22187 64224.86 2.114616 43.52050
7      9 Connecticut 41.26425 1.490933 23.75907 70758.66 2.027202 45.24870
8     10    Delaware 40.94531 1.484375 21.11458 58795.20 2.108073 43.32292
9     11        D.C. 42.76958 1.520152 28.65095 95387.40 1.790875 40.90494
10    12     Florida 41.07563 1.485387 19.53450 54585.91 1.936125 44.37481
11    13     Georgia 40.89145 1.503942 19.82292 55946.86 2.093390 42.54033
12    15      Hawaii 41.21541 1.490368 21.68476 54258.90 2.100701 45.30385
13    16       Idaho 40.85292 1.439648 20.45324 55717.66 2.019984 42.08074
14    17    Illinois 41.09861 1.483566 22.20966 64375.88 2.074701 43.44074
15    18     Indiana 40.77850 1.473941 20.83876 53621.19 2.105863 42.57899
      EMPLR UNEMP_RATE TAX_RTE URB_PER
1  1.137283        6.4  0.0400    59.0
2  1.228374        8.3  0.0000    66.0
3  1.192863        7.8  0.0560    89.8
4  1.154290        6.2  0.0650    56.2
5  1.210346       10.1  0.0725    95.0
6  1.167192        6.8  0.0290    86.2
7  1.178756        7.9  0.0635    88.0
8  1.200521        7.5  0.0000    83.3
9  1.199240        7.9  0.0600   100.0
10 1.143115        8.1  0.0600    91.2
11 1.153426        6.5  0.0400    75.1
12 1.126095       11.7  0.0400    91.9
13 1.105516        5.5  0.0600    70.6
14 1.175299        9.3  0.0625    88.5
15 1.161238        7.3  0.0700    72.4
                              Education_Level    Health_Status Majority_Sex
1  Associate Degree - Occupational/Vocational      Good Health         Male
2  Associate Degree - Occupational/Vocational      Good Health         Male
3  Associate Degree - Occupational/Vocational      Good Health         Male
4  Associate Degree - Occupational/Vocational      Good Health         Male
5  Associate Degree - Occupational/Vocational      Good Health         Male
6         Associate Degree - Academic Program      Good Health         Male
7         Associate Degree - Academic Program      Good Health         Male
8  Associate Degree - Occupational/Vocational      Good Health         Male
9                           Bachelor's Degree Very Good Health       Female
10        Associate Degree - Academic Program Very Good Health         Male
11 Associate Degree - Occupational/Vocational      Good Health       Female
12        Associate Degree - Academic Program      Good Health         Male
13 Associate Degree - Occupational/Vocational      Good Health         Male
14        Associate Degree - Academic Program      Good Health         Male
15 Associate Degree - Occupational/Vocational      Good Health         Male
\end{verbatim}

\newpage

\hypertarget{appendix-c-references}{%
\subsection{Appendix C: References}\label{appendix-c-references}}

\hypertarget{background-1}{%
\subsubsection{Background}\label{background-1}}

\begin{enumerate}
\def\labelenumi{\arabic{enumi}.}
\tightlist
\item
  List your background citations here.
\end{enumerate}

Bureau, U. C. (2021, November 23). Why we conduct the decennial census
of Population and Housing. Census.gov.
\url{https://www.census.gov/programs-surveys/decennial-census/about/why.html\#}:\textasciitilde:text=The\%20results\%20of\%20the\%20census,\%2C\%20roads\%2C\%20and\%20public\%20works.

Mather, M., \& Scommegna, P. (2019, March 15). Why is the U.S. Census so
important?. Population Reference Bureau
\url{https://www.prb.org/resources/importance-of-u-s-census/}

Farley, R. (2020, January 31). The importance of census 2020 and the
challenges of getting a complete count. Harvard Data Science Review.
\url{https://hdsr.mitpress.mit.edu/pub/rosc6trb/release/3}

\hypertarget{data}{%
\subsubsection{Data}\label{data}}

\begin{enumerate}
\def\labelenumi{\arabic{enumi}.}
\tightlist
\item
  List your data citations here
\end{enumerate}

\begin{itemize}
  \item 2020 Unemployment Rates: https://www.bls.gov/lau/lastrk20.htm
  \item 2010 Urban Percentage of Population:https://www.icip.iastate.edu/tables/population/urban-pct-states
  \item 2020 Tax Data: https://taxfoundation.org/data/all/state/2020-sales-taxes/
  \item Download for 2020: https://www.census.gov/data/datasets/2020/demo/cps/cps-asec-2020.html
  \item Codebook for 2020: https://www2.census.gov/programs-surveys/cps/datasets/2020/march/ASEC2020ddl_pub_full.pdf 
\end{itemize}

\end{document}
